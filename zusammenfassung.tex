\chapter{Schlussteil}
\section{Diskussion der Ergebnisse}
Durch die Nutzung von Blockchain-Technologie konnte ein direkter Zahlungskanal zwischen Besitzern der Webseiten und Unternehmen hergestellt werden. Dadurch wird das Schalten von Anzeigen für Besitzer solcher Webseiten profitabler. Gleichzeitig wird mehr Transparenz geschaffen, indem Unternehmen über eine Id direkt auf Metriken, die auf der öffentlichen Blockchain hinterlegt sind, zugreifen können. Somit haben sie einen vollen Einblick in die Situation ihrer Anzeige und können eine beliebige Anzahl an Metriken erfassen. Diese Lösung hat einen positiven Einfluss auf Besucher der Seite, denn ihre Daten werden nicht mehr verwendet und ist in diesem Aspekt datenschutzrechtlich vorzuziehen.\\

Die entwickelte Webanwendung entsprach zwar den Anforderungen zu Zahlungsfluss und Transparenz, weist jedoch gleichzeitig Schwachstellen in diesen Bereichen auf. Die Zahlung zwischen den beiden involvierten Parteien wurde zwar ohne eine Dritte Partei, welche als Zahlungskanal fungiert, ermöglicht, doch liegen, zum Zeitpunkt dieser Arbeit, die Kosten für eine Transaktion laut \cite{etherscan_2021} bei durchschnittlich 2,6 Dollar. Wenn man die Funktionen der Webanwendung betrachtet, stellt dies ein Problem beim Schalten der Anzeigen dar, denn dort wird jedes Mal der Zähler für die Anzeige in der Blockchain erhöht. Jeder Funktionsaufruf im Smart-Contract wird durch eine dementsprechend teure Transaktion ausgelöst. Jede zusätzliche Metrik muss verfolgt werden, sodass die laufenden Kosten für die Internetseite in keinem Verhältnis zum Wert der Impressionen steht.
Die geschaffene Transparenz bringt ebenfalls Nachteile mit sich, denn es kann sich jede Person, mithilfe der Id, Zugriff auf die Metriken von Anzeigen verschaffen. Dies ist aus Sicht der Unternehmen, die miteinander konkurrieren, unvorteilhaft. Zusätzlich zu den genannten Schwachpunkten, verlieren die involvierten Parteien Vorteile des Programmatic Advertising. Durch Verwendung der Blockchain muss die Partei, welche den Werbeplatz zur Verfügung stellt, einen hohen Aufwand betreiben, um die gesamte Infrastruktur bereitzustellen. Im Vergleich dazu, reicht beim Programmatic Advertising das Einfügen eines Code-Snippets in den Quellcode der eigenen Internetseite, was deutlich weniger Programmierkenntnisse erfordert. Den Unternehmen entgeht der Vorteil, dass ihre Anzeigen vollautomatisiert relevante Kundensegmente erreichen.\\

Nach Betrachtung der Ergebnisse kann angenommen werden, dass Online Advertising nicht sonderlich von dem jetzigen Stand der Blockchain-Technologie unter Ethereum profitieren kann.
\section{Zusammenfassung}
Begonnen hat die Arbeit mit der Einleitung...\\
 
Anschließend wurde Blockchain-Technologie am Beispiel von Bitcoin erläutert. Dafür musste allerdings zunächst die Funktionsweise und technische Anforderungen von klassischem Geld untersucht werden, sodass ein Bezug zu Kryptowährung, die auf Blockchain-Technologie basiert, und ihrer Eignung hergestellt werden konnte. 
Danach wurde die Theorie hinter der Blockchain-Technologie am Beispiel von Bitcoin erläutert. 
Das Unterkapitel \emph{Keys und Adressen} folgte dem Weg von der Erzeugung des privaten Schlüssels, welcher Besitzer zum signieren von Transaktionen autorisiert, bis hin zur Generierung von Adressen, mit denen Guthaben empfangen werden kann. 
Daran anknüpfend wurde die Funktionsweise des SHA256-Algorithmus, welcher an verschiedenen Stellen des Bitcoin-Protokolls verwendet wird, nach \cite{dang_2015} erläutert. 
Verglichen wurden anschließend 2 verschiedene Arten von Wallets, die zur Verwaltung von Schlüsseln verwendet werden.
Darauf folgte die Erklärung zu Transaktionen im Bitcoin-Protokoll, die das Prinzip der \emph{Unspent Transaction Outputs} und Transaktionskosten umfasste.
Welche Akteure es im Bitcoin-Protokoll gibt, als auch das Hinzufügen neuer Teilnehmer, wurde in \emph{Die verschiedenen Akteure im Netzwerk} besprochen.
Anschließend folgte das eigentliche Kernthema, die \emph{Blockchain} sowie der Weg, den eine Transaktion bewältigt, bis sie Teil der Blockchain ist. Zum Schluss wird kurz beschrieben, wieso ein Angriff auf das Netzwerk nicht wahrscheinlich ist.\\

Im Kapitel \emph{Blockchain 2.0} wurde das Ethereum-Protokoll näher untersucht. Es wurden Neuerungen im Vergleich zu Bitcoin aufgezeigt, wie andere Mechanismen bei den Transaktionen oder auch der Verbrauch von \emph{Gas}.
Um das neue Konzept der \emph{Smart-Contracts} näher zu untersuchen, wurde zunächst erläutert, wie die \emph{Ethereum Virtual Machine} Anweisungen, die sie in Form von Transaktionen erhält und anhand dieser den Zustand der Blockchain verändert. Daraufhin wurde die Syntax der Programmiersprache \emph{Solidity} aufgezeigt und mit Konzepten gängiger Programmiersprachen verglichen.
Zum Schluss wurden kurz mögliche Use-Cases für den Einsatz von Smart-Contracts genannt.\\

Nachdem ein ausreichender Wissensstand zur Blockchain-Technologie erreicht wurde, beschrieb das Kapitel \emph{Blockchain im Online Advertising} den Kontext der gleichnamigen Industrie. Genauer wurde auf das Thema \emph{Display Advertising} eingegangen und wie heutzutage \emph{Programmatic Advertising} dort Verwendung findet. Die Funktionsweise, sowie Stärken und Schwächen wurden aufgezeigt und anschließend mögliche Verbesserungen mittels Blockchain-Technologie genannt. Im Anschluss daran wurden die Ideen in Form eines Proof-of-Concept umgesetzt und die einzelnen Komponenten, sowie relevanter Programmcode näher erläutert.\\

Es konnte zwar eine funktionsfähige Anwendung entwickelt werden, doch weist diese in anderen Aspekten gravierende Mängel auf, sodass Entschluss gefasst wurde, dass Blockchain-Technologie auf ihrem jetzigen Stand keine Vorteile bietet, die zu einer ernsthaften Alternative zum Programmatic Advertising macht.
\section{Ausblick}
Im Rahmen dieser Arbeit wurden die beiden größten Kryptowährungen Bitcoin und Ethereum näher untersucht, doch sind diese bei Weitem nicht die Einzigen. Rund um das Thema Kryptowährungen hat sich laut \cite{coinmarketcap_2021} eine Industrie mit einer Marktkapitalisierung von derzeit mehr als einer Billion US-Dollar gebildet (Stand Mai 2021). Sogar große US-Banken wie Goldman Sachs nehmen mittlerweile Kryptowährungen wahr, sodass in dieser Industrie weitere Innovationen zu erwarten sind \cite{handelsblatt_2021}. Dies könnten Innovationen wie die Verwendung von Proof-of-Stake statt Proof-of-Work, welches von der Cardano-Plattform verwendet wird und im Paper \cite{kiayias_2019} untersucht wird. Damit stellt vor dem Hintergrund der Klimaerwärmung eine vorteilhafte Alternative zum ressourcen-intensiven Proof-of-Work dar. Genauso könnte das Problem der Skalierbarkeit in Zukunft gelöst werden, um günstigere Transaktionen ohne Verlust der Dezentralisierung ermöglichen zu können.\\

Sobald dadurch günstigere Transaktionen ermöglicht werden, könnten weitere Möglichkeiten von Blockchain-Technologie im Online Advertising untersucht werden. Beispielsweise wird in \cite{ding_2021} ein System vorgestellt, welches, genau wie der Proof-of-Concept dieser Arbeit, Blockchain-Technologie für die Zahlung von Unternehmen, die Werbeflächen für ihre Anzeigen erwerben möchten, verwendet. Zusätzlich dazu wird darin beschrieben, wie man die Nutzer, welche der Werbung ausgesetzt sind, entlohnen könnte. Das Unternehmen \emph{Brave} verfolgt mit ihrem gleichnamigen Browser, welcher einen Ad-Blocker enthält, ein ähnliches Ziel. Nutzer können ihre Zustimmung dafür geben, dass sie dafür entlohnt werden, Werbung, die vom Browser in Form einer Push-Benachrichtigung angezeigt werden, zu erhalten. Mithilfe eines Voting-Systems könnte man Nutzer Anzeigen bewerten lassen, um so, ohne das Sammeln von Daten, wie es beim Programmatic Advertising nötig ist, nur relevante Anzeigen schalten zu können. Dafür würden sie dann anschließend für ihre Aufmerksamkeit eine Entlohnung erhalten. Statt der Sammlung ihrer Daten werden die Nutzer, welche den Anzeigen ausgesetzt sind, somit am Wertschöpfungsprozess beteiligt.


