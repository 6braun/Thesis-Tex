\chapter{Schlussteil}
\section{Diskussion der Ergebnisse}
Durch die Nutzung von Blockchain-Technologie konnte ein direkter Zahlungskanal zwischen Besitzern der Webseiten und Unternehmen hergestellt werden. Dadurch wird das Schalten von Anzeigen für Besitzer solcher Webseiten profitabler. Gleichzeitig wird mehr Transparenz geschaffen, indem Unternehmen über eine Id direkt auf Metriken, die auf der öffentlichen Blockchain hinterlegt sind, zugreifen können. Somit haben sie einen vollen Einblick in die Situation ihrer Anzeige und können eine beliebige Anzahl an Metriken erfassen. Diese Lösung hat einen positiven Einfluss auf Besucher der Seite, denn ihre Daten werden nicht mehr verwendet und ist in diesem Aspekt datenschutzrechtlich vorzuziehen.\\

Die entwickelte Webanwendung entsprach zwar den Anforderungen zu Zahlungsfluss und Transparenz, weist jedoch gleichzeitig Schwachstellen in diesen Bereichen auf. Die Zahlung zwischen den beiden involvierten Parteien wurde zwar ohne eine Dritte Partei, welche als Zahlungskanal fungiert, ermöglicht, doch liegen, zum Zeitpunkt dieser Arbeit, die Kosten für eine Transaktion laut \cite{etherscan_2021} bei durchschnittlich 2,6 Dollar. Wenn man die Funktionen der Webanwendung betrachtet, stellt dies ein Problem beim Schalten der Anzeigen dar, denn dort wird jedes Mal der Zähler für die Anzeige in der Blockchain erhöht. Jeder Funktionsaufruf im Smart-Contract wird durch eine dementsprechend teure Transaktion ausgelöst. Jede zusätzliche Metrik muss auch verfolgt werden, sodass die laufenden Kosten für die Internetseite in keinem Verhältnis zum Wert der Impressionen steht.
Die geschaffene Transparenz bringt ebenfalls Nachteile mit sich, denn es kann sich jede Person, mithilfe der Id, Zugriff auf die Metriken von Anzeigen verschaffen. Dies ist aus Sicht der Unternehmen, die miteinander konkurrieren, unvorteilhaft. Zusätzlich zu den genannten Schwachpunkten, verlieren die involvierten Parteien die Vorteile, welche Programmatic Advertising ihnen bietet. Durch Verwendung der Blockchain muss die Partei, welche den Werbeplatz zur Verfügung stellt, einen hohen Aufwand betreiben, um die gesamte Infrastruktur bereitzustellen. Im Vergleich dazu, reicht beim Programmatic Advertising das Einfügen eines Code-Snippets in den Quellcode der eigenen Internetseite, was deutlich weniger Programmierkenntnisse erfordert. Den Unternehmen entgeht der Vorteil, dass ihre Anzeigen vollautomatisiert relevante Kundensegmente erreichen.\\

Nach Betrachtung der Ergebnisse kann angenommen werden, dass Online Advertising nicht sonderlich von dem jetzigen Stand der Blockchain-Technologie unter Ethereum profitieren kann.
\section{Zusammenfassung der Kapitel}
Nötig?
\section{Ausblick}
Ein 