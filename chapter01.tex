\chapter{Introduction}

\lipsum[2] Meanwhile, information technology~(IT)\nomenclature{IT}{Information technology} has become essential to exchange information between involved inter-organizational actors and across global supply chains more efficiently. Big data can be defined as ``data whose size forces us to look beyond the tried-and-true methods that are prevalent at that time.'' \citep[][p.~44]{jacobs_pathologies_2009}


Although a consistent definition of big data has yet to be specified, there is a common understanding that big data is data whose size and complexity forces us to look beyond conventional tools and methods to exploit and utilize it \citep[cf.][p.~44]{jacobs_pathologies_2009}.

According to \citet[][p.~44]{jacobs_pathologies_2009},  big data can be seen as data whose size $e + p = y$ and complexity forces us to look beyond conventional tool and methods to exploit and utilize it. 

\begin{compactitem}
	\item \citet{fink2006grundlagen} 
	\item \citet{bose2000vehicle} 
	\item \citet{heilig_scientometric_2014} 
	\item \citet{voss_popmusic_2002} 
	\item \citet{davenport_data_2012}
	\item \citet{ropke_heuristic_2005}
	\item \citet{heilig_voss_2015}
\end{compactitem}

\lipsum[2]



\vspace{10pt}
\begin{table}[H]
	\centering
	  \renewcommand{\arraystretch}{1.5}
		\begin{tabular}{p{3.5cm}p{10.5cm}l}
		\toprule
 On-demand \mbox{self-service} & \lipsum[1]
 \\\midrule
 Elasticity \mbox{and scalability} & \lipsum[1]
\\\bottomrule
 \end{tabular}
	\caption[Key characteristics of cloud computing]{Key characteristics of cloud computing \citep{armbrust2010view}}
	\label{tab:cloud.characteristics}
	
\end{table}