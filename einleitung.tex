\chapter{Einleitung}
Im Jahr 1956 wurde die erste Werbung im deutschen Fernsehen ausgestrahlt \cite{tagesspiegel_2016}. Mit ihrem Werbespot legte das Unternehmen \emph{Henkel} den Grundstein für eine Industrie, die Investitionen von 26,79 Milliarden Euro im Jahr 2018 rechtfertigt \cite{statista_werbung_2020}. Durch Innovationen wie das Internet und Smartphones verbringt laut \cite{statista_internetnutzung_2020} ein Großteil der Menschen in Deutschland heute viel Zeit im digitalen Raum und dementsprechend dringt auch Werbung in diesen ein. Für das automatisierte Vermitteln von Anzeigen kümmern sich sogenannte \emph{Ad-Broker}, die gleichzeitig auch als Zahlungskanal zwischen den involvierten Parteien fungieren. Also Folge dessen mangelt es dem Vergabeprozess an Transparenz und eine Provision wird von ihnen erhoben.\\

In den letzten Jahren haben Kryptowährungen eine erhöhte Aufmerksamkeit bekommen und gewinnen immer mehr an Akzeptanz, jedoch werden diese oft nur als spekulative Investitionsmöglichkeit betrachtet. Hinter ihnen verbirgt sich allerdings die sogenannte \emph{Blockchain-Technologie}, die als Lösung für Zahlung und Transparenz dienen könnte. Aufgrund ihrer Möglichkeiten stellt sich die Frage, inwieweit das Online Advertising vom heutigen Stand der Blockchain-Technologie profitieren kann.\\

Im Rahmen dieser Arbeit soll den Leser:innen ein Einblick in das Online Advertising verschafft werden. Die derzeitige Funktionsweise und Schwachpunkte des Systems sollen aufgezeigt werden, sodass mögliche Lösungsansätze mithilfe von Blockchain-Technologie gefunden werden können. Insbesondere soll den Leser:innen ein Verständnis über Blockchain-Technologie vermittelt werden, das über den Charakter einer rein spekulativen Investition hinausgeht.\\

Die Bachelorarbeit besteht aus drei Kapiteln. Im Ersten wird das derzeitige Finanzsystem thematisiert und sowie mögliche Vorteile von Blockchain-Technologie genannt. Im Anschluss wird die grundlegende Technologie am Beispiel von \emph{Bitcoin} erläutert. Das zweite Kapitel beschäftigt sich damit, wie \emph{Ethereum} bereits thematisierte Konzepte erweitert. Sobald ein ausreichendes Wissen über Blockchain-Technologie erreicht wurde, wird im dritten Kapitel das Online Advertising dargestellt und in Form eines Proof-of-Concept eine geeignete Webanwendung, welche Blockchain-Technologie verwendet, entwickelt. Den Schluss bilden eine Diskussion der Ergebnisse sowie Beantwortung der Leitfrage, eine Zusammenfassung der Kapitel und ein Ausblick, in dem mögliche kommende Innovationen genannt und alternative Anwendungsbeispiele genannt werden. 

