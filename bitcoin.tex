\chapter{Blockchain 1.0 - Wie Bitcoin funktioniert}
Mit Fortschritt im Bereich Kryptographie begann auch das Interesse von Forschern an digitalen Währungen. 
Das Problem dieser frühen Projekte bestand jedoch darin, dass sie, einen sogenannten \emph{Central Point of Failure}, also eine zentralisierte Schwachstelle besaßen. 
Beispielsweise könnten die Konten von Nutzern zwar kryptografisch gesichert, jedoch von zentralen Stellen wie Banken verwaltet werden.\\
Ein wichtiges Problem, welches es mithilfe von Geld zu lösen gilt, ist das sogenannte \emph{Double Spending Problem}. Es muss durch gewisse Mechanismen verhindert werden, dass bösartige Akteure die selben Geldwerte für mehrere Transaktionen verwenden. Bei physischem Geld, also Geldscheinen, Münzen, etc. verhindern komplexe Drucktechniken die Verbreitung von Falschgeld und dadurch das ein Geldschein nur einmal existieren kann, ist dieser nur für eine Transaktion zu verwenden.\\
Versucht man nun diese Geldwerte gänzlich digital zu verwalten, so liegt die Verantwortung für eine korrekte Beobachtung und Verwaltung bei einer zentralen Stelle wie einer Bank. Diese könnte als Angriffsstelle für Antagonisten dienen und stellt somit eine Gefahr für das System dar.\\
Dieses Kapitel beschäftigt sich mit der traditionellen Funktionsweise von Geld und wie mithilfe eines dezentralen Systems ein zentraler Fehlerpunkt vermeidet werden kann. 
\section{Funktionsweise einer klassischen Transaktion von Geld}
Eine Währung, die zum Verwalten und Versenden von monetärem Wert dient, hat drei Probleme zu lösen:
\begin{enumerate}
	\item Sicherstellung des Wertes, also die Authentizität
	\item Garantie dafür, dass die selbe Währung nicht mehr als einmal verwendet werden kann (Double Spending)
	\item Zugang zur Währung nur für befugten Besitzer
\end{enumerate}
TODO
\section{Notwendigkeit für Blockchain-Technologie}
\section{Theorie der Blockchain-Technologie am Beispiel von Bitcoin}
Auch wenn es andere Projekte für dezentrale Währungen wie B-Money und Hashcash gab, begann der Aufschwung digitaler Währungen im Jahr 2008 mit der Veröffentlichung des Bitcoin-Whitepapers \emph{Bitcoin: A Peer to peer Electronic Cash System}. Diese Publikation wurde, von einer bis heute unbekannten Person, unter dem Namen \emph{Satoshi Nakamoto} veröffentlicht und kombinierte Technologien von seiner Vorgänger. Statt einer zentralen Verwaltungsstelle handelt es sich bei Bitcoin um ein dezentrales Peer-to-peer Netzwerk zwischen den Nutzern des Bitcoin-Protokols. Außerdem werden Vermögenswerte nicht durch klassischer Münzen auf einem Konto repräsentiert, sondern durch vergangene Transaktionen in einem dezentralen und öffentlichen Transaktionsbuch, dem sogenannten \emph{Ledger} impliziert. Aufgrund dieser Eigenschaften besteht keine zentrale Angriffsfläche für bösartige Akteure und jeder Akteur im Netzwerk hat Kenntnis über alle Transaktionen. Die folgenden Untersektionen beschäftigen sich mit der Verwaltung und dem Zugang für Nutzer, die Funktionsweise von Transaktionen sowie die Art und Weise, wie die verschiedenen Akteure im Netzwerk zu einem gemeinsamen Konsens kommen.
\subsection{Keys und Adressen}
\subsection{Transaktionen}
\subsection{Zeitstempel}
\subsection{Der Konsensalgorithmus Proof-of-Work}
\subsection{Netzwerk}
\subsection{Anreize}
\subsection{Freimachen von Speicherplatz mittels Hash-Bäumen}
\subsection{Verifizierung von Transaktionen}
\subsection{Sicherheit und Privatsphäre}
\subsection{Angriff auf das Netzwerk}

