\chapter{Blockchain 1.0 - Wie Bitcoin funktioniert}
Mit Fortschritt im Bereich Kryptographie begann auch das Interesse von Forschern an digitalen Währungen. 
Das Problem dieser frühen Projekte bestand jedoch darin, dass sie einen sogenannten \emph{Central Point of Failure}, also eine zentralisierte Schwachstelle besaßen. 
Beispielsweise könnten die Konten von Nutzern zwar kryptografisch gesichert, jedoch von zentralen Stellen wie Banken verwaltet werden müssen.\\
Ein wichtiges Problem, welches es mithilfe von Geld zu lösen gilt, ist das sogenannte \emph{Double Spending Problem}. Es muss durch gewisse Mechanismen verhindert werden, dass bösartige Akteure die selben Geldwerte für mehrere Transaktionen verwenden. Bei physischem Geld, also Geldscheinen, Münzen, etc. verhindern komplexe Drucktechniken die Verbreitung von Falschgeld und dadurch das ein Geldschein nur einmal existieren kann, ist dieser nur für eine Transaktion zu verwenden.\\
Versucht man nun diese Geldwerte gänzlich digital zu verwalten, so liegt die Verantwortung für eine korrekte Beobachtung und Verwaltung bei einer zentralen Stelle wie einer Bank. Diese könnte als Angriffsstelle für Antagonisten dienen und stellt somit eine Gefahr für das System dar.\\
Dieses Kapitel beschäftigt sich mit der traditionellen Funktionsweise von Geld und wie mithilfe eines dezentralen Systems ein zentraler Fehlerpunkt vermeidet werden kann. 
\section{Funktionsweise von Geld}
\cite{mankiw_taylor_2018} bezeichnen Geld als ein Bündel von Aktiva, das die Menschen in einer Volkswirtschaft regemäßig dazu verwenden, Waren und Dienstleistungen von anderen Menschen zu erwerben.\\
Es erlaubt den Parteien einem Tauschgeschäft, bei dem beide Seiten mit dem Gut des Tauschpartners zufrieden sein müssen, zu entgehen und ermöglicht stattdessen eine effiziente Allokation von Ressourcen. Zugleich stellt Geld sicher, dass das eigene Kapital den Wert auch in Zukunft behält.\\
Damit ein Handelsgut als Geld angesehen werden kann, muss es drei Funktionen erfüllen können\\
Fundamental ist, dass das Handelsgut generell als Tausch- bzw. Zahlungsmittel akzeptiert wird. Theoretisch könnte man versuchen sein Abendessen mit dem eigenen Fahrrad zu bezahlen, doch kommt man in der Praxis mit dieser Strategie nicht weit.\\
Des Weiteren muss das Tauschgeschäft als Recheneinheit fungieren können. Dies ist notwendig, da anhand Dessen die relativen Preise anderer Waren in der Marktwirtschaft ermittelt werden müssen.\\
Zuletzt muss sichergestellt sein, dass das Handelsgut wie bereits erwähnt in Zukunft auch seine Kaufkraft behält. Jemand, der es als Zahlungsmittel akzeptiert muss sich darauf verlassen können, dass es auch für zukünftige Geschäfte verwendet werden kann.\\
Bei den Geldformen unterscheidet man zwischen Warengeld und Rechengeld. Diese unterscheiden sich in ihrem intrinsischen Wert, also darin, ob sie auch außerhalb von Tauschgeschäften einen Nutzen finden. Ein Beispiel für Warengeld ist Gold, welches neben Tauschgeschäften auch industriell verarbeitet werden kann.\\
Papiergeld hingegen bietet abseits des Tauschgeschäftes keinen Nutzen für den Besitzer. Um trotzdem den Wert des Geldes gewährleisten zu können, wird es von Seiten des Staats als universelles Zahlungsmittel in der jeweiligen Marktwirtschaft bestimmt.\\
Eine weitere wichtige Rolle im Finanzsystem nehmen Zentralbanken ein. Sie überwachen das Bankensystem und steuern über eine geeignete Geldpolitik das Geldangebot auf dem Markt.\\
Durch das Drucken von Geld und den anschließenden Kauf von Wertpapieren können sie das Geldangebot erhöhen. Um es wiederum zu verringern, verkaufen sie Wertpapiere und nehmen das erhaltene Geld aus dem Umlauf.\\

Eine Währung, die zum Verwalten und Versenden von monetärem Wert dient, hat drei technische Anforderungen zu erfüllen:
\begin{enumerate}
	\item Sicherstellung des Wertes, also die Authentizität
	\item Garantie dafür, dass die selbe Währung nicht mehr als einmal verwendet werden kann (Double Spending)
	\item Zugang zur Währung nur für befugten Besitzer
\end{enumerate}
TODO
\section{Notwendigkeit für Blockchain-Technologie}
\section{Theorie der Blockchain-Technologie am Beispiel von Bitcoin}
Auch wenn es andere Projekte für dezentrale Währungen wie B-Money und Hashcash gab, begann der Aufschwung digitaler Währungen im Jahr 2008 mit der Veröffentlichung des Bitcoin-Whitepapers \emph{Bitcoin: A Peer to peer Electronic Cash System}. Diese Publikation wurde, von einer bis heute unbekannten Person, unter dem Namen \emph{Satoshi Nakamoto} veröffentlicht und kombinierte Technologien ihrer Vorgänger. Statt einer zentralen Verwaltungsstelle handelt es sich bei Bitcoin um ein dezentrales Peer-to-peer Netzwerk zwischen den Nutzern des Bitcoin-Protokols. Außerdem werden Vermögenswerte nicht durch klassischer Münzen auf einem Konto repräsentiert, sondern durch vergangene Transaktionen in einem dezentralen und öffentlichen Transaktionsbuch, dem sogenannten \emph{Ledger} impliziert. Aufgrund dieser Eigenschaften besteht keine zentrale Angriffsfläche für bösartige Akteure und jeder Akteur im Netzwerk hat Kenntnis über alle Transaktionen. Die folgenden Untersektionen beschäftigen sich mit der Verwaltung und dem Zugang für Nutzer, die Funktionsweise von Transaktionen sowie die Art und Weise, wie die verschiedenen Akteure im Netzwerk zu einem gemeinsamen Konsens kommen.
\subsection{Keys und Adressen}
Als Kryptographie bezeichnet man Verfahren zur Verschlüsselung von Informationen, die schon von den Nazis im zweiten Weltkrieg genutzt wurden. 
Mithilfe von Maschinen, den sogenannten \emph{ENIGMA}, verschlüsselten sie wichtige strategische Informationen wie die Aufenthaltsorte von Truppen oder taktische Befehle, die anschließend per Funk überbracht wurden.\\
Kryptographische Verfahren folgten zu der Zeit dem Prinzip \emph{Security by Obscurity}, nach dem die Sicherheit eines Verschlüsselungsverfahrens davon abhängig ist, ob die Funktionsweise dieser bekannt ist. 
Dies hatte zur Folge, dass im Falle der Nazis, deren ENIGMA-Code im Jahr 1941 vom englischen Mathematiker \emph{Alan Turing} und seinem Team gelöst werden konnte.\\
Im Jahr 1976 stellten \emph{Diffie} und \emph{Hellman} die bis dahin unbekannte asymmetrische Verschlüsselung vor, bei der jede Partei ein Schlüsselpaar, bestehend aus privatem und öffentlichem Schlüssel, besitzt. 
Derartige Verfahren sind heutzutage der Standard und werden auch im Bitcoin-System verwendet.\\
Für Bitcoin wird ein Paar aus Schlüsseln erzeugt. 
Dieses Paar besteht aus dem privaten Schlüssel (private key), welcher nur dem Besitzer bekannt ist und zum Signieren von Transaktionen nötig ist.
Aus diesem wird durch die Verwendung von Hashing-Verfahren ein öffentlicher Schlüssel (public key) abgeleitet, mit dem Bitcoins empfangen werden können.\\
Außerdem kann aufgrund der mathematischen Abhängigkeit zwischen den Schlüsseln eine durch den privaten Schlüssel signierte Transaktion mithilfe des öffentlichen Schlüssels verifiziert werden. Dies geschieht, indem der Absender die Transaktion mit seinem privaten Schlüssel signiert und die Authentizität der Signatur mithilfe des öffentlichen Schlüssels von anderen Akteuren des Netzwerks verifiziert wird. Um Begünstigter einer Transaktion zu sein, muss man eine Adresse besitzen und diese an andere Nutzer des Netzwerks propagieren. Um Jene zu erzeugen, wird der öffentliche Schlüssel genutzt, welchen man nicht wieder aus der Adresse rekonstruieren kann.\\
\begin{figure}[htpb]
	\centering
	\includegraphics[width=\textwidth]{images/public_and_private_key.png}
	\caption{Generierung der Schlüssel bzw. Adressen aus dem jeweiligen Vorgänger}
	\label{6braun:fig:keys}
\end{figure}
\subsubsection{Private Keys}
Ein privater Schlüssel besteht aus einer Zahl von 256 zufälligen Bits. Er wird zum Signieren von Transaktionen und für den Zugriff auf ein Guthaben benötigt. Ohne privaten Schlüssel verliert man als Besitzer von Bitcoin auch den Zugriff auf das eigene Guthaben.\\
Um einen privaten Schlüssel generieren zu können, benötigt man eine sichere Quelle für "Zufälligkeit". In anderen Worten: Die Wahl der zufälligen Zahl darf nicht vorhersehbar sein. Dazu verwendet die Bitcoin-Software den Random Number Generator des verwendeten Betriebssystems kombiniert mit einem menschlichen Input, wie dem Bewegen der Maus. Mithilfe des Generators erzeugt man einen zufälligen String, welcher \emph{mehr} als 256 Bits hat. Diesen lässt man anschließend durch den SHA256 Hash-Algorithmus laufen und prüft, ob die resultierende Zahl kleiner ist, als die vom Bitcoin-Protokol gewählte Konstante \emph{n} ($n = 1.1578 * 10^{77}$).

\subsubsection{Public Keys}
Um einen öffentlichen Schlüssel aus dem Privaten generieren zu können, benötigt man ein kryptografisches Verfahren, welches eine Rekonstruktion des Privaten aus dem öffentlichen Schlüssel nicht zulässt.
Das vom Bitcoin-Protokol verwendete Verfahren wird \emph{Elliptic Curve Cryptography} gennant und bedient sich an den Eigenschaften einer Ellipse.
\begin{figure}[htpb]
	\centering
	\includegraphics[width=0.7\textwidth]{images/elliptic_graph_cryptography.png}
	\caption{Die von Bitcoin verwendete Ellipse mit der Funktion $y^{2} = x^{3} + 7$ \\TODO: Bessere Grafik}
	\label{6braun:fig:ellipse}
\end{figure}
Um einen öffentlichen Schlüssel zu generieren, wählt man einen Punkt, den sogenannten Generatorpunkt, auf der Ellipse und Multipliziert diesen mit dem vorher generierten privaten Schlüssel. Eine Multiplikation kann auch als Addition einer Zahl mit derselben betrachtet werden. Um den Punkt G auf der Ellipse mit sich selbst zu addieren, zieht man an diesem die Tangente und berechnet den Schnittpunkt von Ellipse und der gezogenen Tangente. Anschließend spiegelt man den Punkt an der x-Achse, erhält 2G. Diese Addition führt man so oft aus, wie der 256 Bit lange private Schlüssel groß ist, sodass man am Ende einen Punkt (x,y) erhält, welcher als öffentlicher Schlüssel genutzt werden kann. Diesen generierten Schlüssel kann man veröffentlichen, denn aus ihm lässt sich nicht schließen, mit welchem Faktor der Generatorpunkt multipliziert wurde.

\subsubsection{Bitcoin Adressen}
Eine Adresse ist ein aus dem öffentlichen Schlüssel generierter String aus Buchstaben und Zahlen, der den Besitzer des Schlüssels zu einem potentiellen Empfänger einer Transaktion macht. Beim diesem muss es sich allerdings nicht zwangsläufig um eine Person handeln, denn auch Organisationen, geschriebene Skripte, etc. kommen als \emph{abstrakter} Empfänger in Frage.\\
So wie der Öffentliche aus dem privaten Schlüssel erzeugt wird, wird die Bitcoin Adresse aus dem öffentlichen Schlüssel mithilfe von Hashing-Algorithmen erzeugt. 
Die verwendeten Algorithmen, welche nacheinander auf den öffentlichen Schlüssel angewendet werden, heißen SHA-256 und RIPEMD160.\\
Da Verschlüsselungsalgorithmen einen Grundbaustein für Blockchain-Technologie darstellen, wird ihre Funktionsweise im Folgenden beispielhaft anhand des SHA-256-Algorithmus erläutert. 

\subsection{Einschub: SHA-256}
Ein Hashalgorithmus ist eine mathematische Funktion, die einen Input entgegennimmt und einen Output fester Größe, den sogenannten Hashwert erzeugt. Ein Algorithmus sollte idealerweise folgende Eigenschaften bieten:
\begin{itemize}
	\item Für einen gegebenen Input immer denselben Output generieren
	\item Nur in eine Richtung berechenbar sein
	\item Durch geringe Änderungen am Input einen völlig anderen Output generieren
\end{itemize}

Hashfunktionen haben vielerlei Anwendungsmöglichkeiten im Bereich der Datensicherheit und eine verwendete Hashfunktion im Bitcoin-Protokoll ist die bereits erwähnte SHA256-Funktion, welche u.A. von der NSA entwickelt wurde. Die folgende Erläuterung des Algorithmus orientiert sich an \cite{dang_2015}, ist allerdings für das Verständnis nachfolgender Kapitel nicht zwingend erforderlich.\\

\subsubsection{Definitionen}
Bevor der Algorithmus erläutert werden kann, ist es notwendig folgende Operationen, sowie benötigte Konstanten zu definieren. Die Operationen werden auf Zahlen in Binärform bitweise angewendet.
\begin{longtable}{p{3.5cm}p{10.5cm}l}
\caption{Benötigte Operationen}
	
		\\\toprule
		XOR  & Vergleichender Operator $\oplus$. Ergebnis ist True(1), wenn genau einer der beiden Inputs True ist. Andernfalls wird False(0) ausgegeben. Beispiel: $$0 \oplus 1 = 1$$
		\\\midrule
		Rotation Right  & Verschiebung der Bits um n Stellen nach rechts. Bei Overflow werden Bits wieder vorne angefügt. Beispiel: $$ROTR_1(011) = 101$$
		\\\midrule
		Shift Right  & Ähnlich wie die Rotation, nur dass Bits an letzter Stelle wegfallen und eine Nullen vorne angefügt werden. Beispiel: $$SHR_1(011) = 001$$
		\\\midrule
		Choice  & Nimmt drei Zahlen gleicher Länge entgegen und entscheidet anhand der Bits des ersten Parameters, welches Bit der jeweils anderen Parameter übernommen werden soll. $$Ch(x,y,z) = (x \land y)\oplus(\lnot x \land z)$$
		Beispiel: $Ch(110,001,101) = 001$
		\\\midrule
		Majority  & Nimmt drei Zahlen gleicher Länge entgegen und übernimmt an jedem Bit denjenigen Wert, der zwischen den Inputs am häufigsten auftaucht.
		$$Maj(x,y,z) = (x \land y)\oplus(x \land z)\oplus(y \land z)$$
		Beispiel $Maj(110,001,101) = 101$
		\\\midrule
		$\sigma_0$  & $\sigma_0$ und die folgenden Funktionen sind definierte Folgen der oben definierten Operationen.
		$$\sigma_0(x) = ROTR_7(x) \oplus ROTR_{18}(x) \oplus SHR_3(x)$$ 
		\\\midrule
		$\sigma_1$  & $$\sigma_1(x) = ROTR_{17}(x) \oplus ROTR_{19}(x) \oplus SHR_10(x)$$
		\\\midrule
		$\Sigma_0$  & $$\Sigma_0(x) = ROTR_2(x) \oplus ROTR_{13}(x) \oplus ROTR_22(x)$$
		\\\midrule
		$\Sigma_1$  & $$\Sigma_1(x) = ROTR_6(x) \oplus ROTR_{11}(x) \oplus ROTR_{25}(x)$$
		\\\bottomrule
\end{longtable}
Außerdem müssen noch zwei Listen mit Konstanten definiert werden. Die Liste K beinhaltet die ersten 32 Bit der Nachkommastellen der Kubikwurzeln der ersten 64 Primzahlen.
H die ersten 32 Bit der Nachkommastellen der Quadratwurzeln der ersten 8 Primzahlen. 
 Die genauen Werte haben keine sonderliche Bedeutung, sondern lediglich die scheinbare Zufälligkeit ist von Relevanz. Indem man diese berechenbaren Zahlen nimmt, minimiert sich die Wahrscheinlichkeit für eine mögliche Backdoor im Algorithmus.
\begin{lstlisting}[caption={Liste K von Konstanten},captionpos=b]
	428a2f98 71374491 b5c0fbcf e9b5dba5 3956c25b 59f111f1 923f82a4 
	ab1c5ed5 d807aa98 12835b01 243185be 550c7dc3 72be5d74 80deb1fe 
	9bdc06a7 c19bf174 e49b69c1 efbe4786 0fc19dc6 240ca1cc 2de92c6f 
	4a7484aa 5cb0a9dc 76f988da 983e5152 a831c66d b00327c8 bf597fc7
	c6e00bf3 d5a79147 06ca6351 14292967 27b70a85 2e1b2138 4d2c6dfc
	53380d13 650a7354 766a0abb 81c2c92e 92722c85 a2bfe8a1 a81a664b 
	c24b8b70 c76c51a3 d192e819 d6990624 f40e3585 106aa070 19a4c116 
	1e376c08 2748774c 34b0bcb5 391c0cb3 4ed8aa4a 5b9cca4f 682e6ff3 
	748f82ee 78a5636f 84c87814 8cc70208 90befffa a4506ceb bef9a3f7 
	c67178f2
\end{lstlisting}
\begin{lstlisting}[caption={Liste H mit den Arbeitsvariablen H0 - H7},captionpos=b]
	H0 = 6a09e667
	H1 = bb67ae85
	H2 = 3c6ef372
	H3 = a54ff53a
	H4 = 510e527f
	H5 = 9b05688c
	H6 = 1f83d9ab
	H7 = 5be0cd19
\end{lstlisting}

\subsubsection{Vorbereitung der Nachricht}
Bevor eine Nachricht verarbeitet werden kann, muss sie in eine für den Algorithmus brauchbare Form gebracht werden. 
Da der SHA-256 mit Zahlen in Binärform arbeitet, müssen Inputs in Stringformat über die ASCII-Tabelle in Binärzahlen umgewandelt werden. So wird z.B. aus dem String 'ab' die Zahl 01100001 01100010.
Anschließend fügt man eine Eins und k-viele Nullen hinzu, dass ein Vielfaches von 512 abgezogen 64 herauskommt. \cite{dang_2015} definiert diese Voraussetzung durch
$$l+1+k \equiv 448 Mod512$$
mit l als Länge der Nachricht. Anschließend werden 64 Bits hinzugefügt, welche die Zahl l repräsentieren, im Fall 'ab' wäre das die 16 in 64-Bit Repräsentation. Die resultierende N*512-Bit Zahl stellt den Input für die eigentliche Verschlüsselung dar.

\subsubsection{Aufteilung der Nachricht in Blöcke}
Die resultierende Zahl wird anschließend in Nachrichtenblöcke $M_1$ - $M_N$ der Länge 512-Bit und diese wiederum in 16 32-Bit-Blöcke $M_{i0}$ - $M_{i15}$ aufgeteilt. Der String 'ab' resultiert in einer 512-Bit-Zahl, sodass an dieser Stelle nur eine Aufteilung in 16 32-Bit-Blöcke nötig wäre.

\subsubsection{Algorithmus}
\begin{lstlisting}[mathescape]
	message = ''
	For i=1 to N:
		w = []
		For t=0 to 63:
			if t <= 15:
				w[i].push(M[i][t])
			else:
				a = $\sigma_1 (w[t-2]) + w[t-7] + \sigma_0 (w[t-15]) + w[t-16]$
		a = H0
		b = H1
		c = H2
		...
		h = H7
		For j=0 to 63
			T1 = h + $\Sigma_1$(e) + Ch(e,f,g) + k[j] + w[j]
			T2 = $\Sigma_0$(a) + Maj(a,b,c)
			h = g
			g = f
			e = d + T1
			d = c
			c = b
			b = a
			a = T1+T2
		H0 = a + H0
		H1 = b + H1
		...
		H7 = h + H7
		result = ''.concat(H1).concat(H2). ... .concat(H7)
		message = message.concat(result)
\end{lstlisting} 
\subsection{Wallet}
Eine Wallet ist ein Programm, welches als Interface zwischen Bitcoin-Netzwerk und dem Nutzer dient. 
Dessen Funktionen beinhalten die Verwaltung der Schlüssel, das Berechnen des Guthabens und das Signieren von Transaktionen.
Eine Wallet ist, im Gegensatz zu einer physikalischen Geldbörse nicht für das Halten von Münzen, sondern zur Verwaltung der privaten Schlüssel zuständig. 
Wie das Berechnen des "Guthabens" geschieht, wird im Unterkapitel \emph{Transaktionen erläutert}.\\
Man unterscheidet zwischen nicht-deterministischen und deterministischen Wallets. Die erste Variante kann man sich als Korb vorstellen, in dem vorher zufällig generierte private Schlüssel in großer Anzahl gelagert sind. Dabei erzeugt ein Privater einen öffentlichen Schlüssel, der wiederum eine Adresse erzeugt (siehe Kapitel \emph{Keys und Adressen}).
Um die eigene Pseudonymität zu schützen ist es empfehlenswert, einen Key nur ein Mal zu benutzen. Aufgrund der hohen Anzahl angesammelter Keys und der damit verbundenen Datensicherung ist diese Art Wallet heute nicht mehr der Standard.\\
\begin{figure}[htpb]
	\centering
	\includegraphics[width=0.7\textwidth]{images/non_det_wallet.png}
	\caption{Nicht-deterministische Wallet}
	\label{6braun:fig:non-deterministic_wallet}
\end{figure}
Die fortgeschrittenste Form einer Deterministischen ist die sogenannte BIP32-Wallet, welche 2 nützliche Eigenschaften aufweisen kann (BIP steht für \emph{Bitcoin Improvement Proposal} und bezeichnet eine nachträgliche Ergänzung zum Bitcoin-Ökosystem).\\
Diese werden in \cite{buterin_2013} als \emph{Master Public Key Property} und \emph{Hierarchy Property} bezeichnet.
Die Master Public Key Property beschreibt die Möglichkeit, aus einem Master Private einen Master Public Key zu generieren, der wiederum alle öffentlichen Schlüssel und deren Adressen erzeugen kann. Dazu berechnet man den sogenannten Offsets, indem man den gewünschten Index und den Master Public Key addiert und das Ergebnis als Input für eine Hashfunktion verwendet. Anschließend addiert man Offset und Master Public Key und erhält den öffentlichen Schlüssel am Index.
$$offset = SHA256(index + masterPubKey)$$
$$pubKey_{index} = offset + masterPubKey$$ Dies geht analog genauso mit dem Master Private Key. 
Aufgrund dieser Eigenschaft ist es möglich, den Master Public Key ungeschützt zu lagern und sogar an dritte Parteien herauszugeben, ohne dass diese Zugriff auf das Guthaben erhalten.\\
Die Hierarchieeigenschaft wird im Kontext einer Organisation mit verschiedenen Organisationszweigen interessant. Ein Geschäftsführer könnte so den unterschiedlichen Geschäftszweigen seines Unternehmens Schlüsselpaare zuweisen, wodurch diese die Verfügungsgewalt über das Eigene und Guthaben von Unterstellen erhalten. Gleichzeitig behält der Geschäftsführer die absolute Kontrolle über alle Schlüssel, da er im Besitz der Master Keys ist. 
\begin{figure}[htpb]
	\centering
	\includegraphics[width=0.7\textwidth]{images/bip32_wallet.png}
	\caption{BIP32-Wallet}
	\label{6braun:fig:non-deterministic_wallet}
\end{figure}
Anders als bei einer nicht-deterministischen Wallet müssen nicht mehr die Private Keys selbst, sondern lediglich der \emph{Seed} gesichert werden. Dieser ist seit dem BIP39/44 eine kurze Liste von für den Menschen leserlichen Worten. Diese können mithilfe eines speziellen Dictionaries in eine Hex-Zahl umgewandelt werden, aus der schließlich der Master Private Key erzeugt wird.

\subsection{Transaktionen}
\subsection{Netzwerk}
\subsection{Der Konsensalgorithmus Proof-of-Work}
\subsection{Angriff auf das Netzwerk}

